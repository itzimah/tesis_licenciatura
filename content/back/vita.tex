\chapter{Ap\'endices}

\section{C\'alculo del factor de correccion del PMT}\label{apendicitis} 
 Cuando la fluorescencia de la muestra de inter\'es y de la referencia (Rodamina 6G) tienen longitudes de onda similares, el PMT responder\'a de la misma manera en ambas muestras; sin embargo, si la muestra y la referencia emiten a diferentes longitudes de onda no tendr\'an la misma respuesta 







 es necesario introducir un factor de correci\'on.


\footnote{Este factor de correcci\'on se obtiene de las curvas de respuesta del PMT contenidas el manual de dicho equipo} 







\clearpage
\thispagestyle{empty}
