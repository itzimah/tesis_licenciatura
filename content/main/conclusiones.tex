\chapter{Discusi�n de resultados y conclusiones}   %\label{chap_content}


Los valores de $\sigma^{TPA}$ que se obtuvieron para el copol�mero PF2/6-b-P3TMAHT resultaron muy bajos en comparaci�n con otros pol�meros del mismo tipo de composici�n (basados en politiofenos y polifluorenos). Cabe mencionar que la \'unica soluci�n cuya emisi�n no lineal pudo ser detectada, utilizando el dise�o experimental que se present� en la secci�n \ref{laserlau}, fue la soluci�n de alta concentraci�n $c=9.25\times 10^{-6}M$; es posible que a esta concentraci�n se formen estructuras de tipo lamerar, provocando grandes aglomerados e interacciones que provoquen una extinci\'on o autoextinci�n de fluorescencia (Quenching o self-quenching en ingles). Tambi�n es posible que se formen estructuras lamelares, no precisamente por la concentraci�n, sino porque el copol�mero est� compuesto por dos bloques que tienen diferente solubilidad ante distintos solventes, particularmente agua y THF, y en la muestra que se estudi� se diluy� el copol�mero en un volumen de agua-THF en proporci�n 1:1.


%Cuando el copol�mero se disuelve en agua, es 




Uno de los materiales de mayor inter\'es que se estudiaron en este trabajo fue la mol\'ecula octopolar, que a pesar de tener una baja eficiencia cu\'antica de fluorescencia, present\'o una absorci\'on no lineal m\'as grande que las mol\'eculas dipolar y cuadrupolar, indicando que la no linealidad aumenta con la complejidad de la distribuci\'on de cargas en una mol\'ecula. La suspensi\'on acuosa de nanopart\'iculas funcionalizadas con PEG present\'o transparencia y estabilidad; la suspensi\'on ha sido expuesta a la luz y calor del ambiente en un periodo de ocho meses desde su preparaci\'on y no se ha manifestado una precipitaci\'on del material. 



transferencia de energiiia

%las nps de las laminillas siguieron fluoresciendo 


En relaci\'on con la internalizaci\'on de nanopart\'iculas, se realizaron pruebas con la suspensi\'on de nanopart\'iculas de PMC300 dopadas con Rodamina 6G, utilizando c\'elulas sin permeabilizar. En este caso se observ\'o una distribuci\'on irregular o no uniforme de nanopart\'iculas en las c\'elulas (en comparaci\'on con la distribuci\'on en c\'elulas permeabilizadas) y se form\'o una cantidad grande de aglomerados en la membrana celular, no se observaron nanopart\'iculas en el citoplasma. Por otra parte, se observ\'o un contraste mayor entre las c\'elulas y las nanopart\'iculas utilizando una sonda sin suero fetal bovino, es decir, exponiendo las c\'elulas \'unicamente a la suspensi\'on acuosa de nanopart\'iculas.


En las im\'agenes adquiridas con el microscopio de epifluorescencia para la mol\'ecula octopolar y el pol\'imero PMC300 logr\'o observarse la tinci\'on celular tanto del DAPI en los n\'ucleos como de las nanopart\'iculas en el citoplasma de las c\'elulas; sin embargo, como trabajo futuro se contempla la obtenci\'on de im\'agenes de las nanopart\'iculas en las l\'ineas celulares utilizando un microscopio multifot\'on, espec\'ificamente de absorci\'on de dos fotones.           

El desarrollo de este trabajo involucr\'o un aprendizaje b\'asico para el manejo de diversos dispositivos y equipo especializado del Centro de Investigaciones en \'Optica (CIO), tales como l\'aseres pulsados, amplificadores, detectores, microscopio de epifluorescencia, etc. De igual manera hubo un aprendizaje de t\'ecnicas y m\'etodos, dentro de algunos laboratorios, para fabricar soluciones, nanopart\'iculas, realizar funcionalizaciones y todos los preparaciones que aqu\'i se mencionan, utilizando materiales y reactivos qu\'imicos proporcionados por el CIO.  









%Para medir eficiencia cu\'antica de fluorescencia, en ocasiones no se contaba con el amplificador lock- in y se realizaron pruebas utilizando un mult\'imetro.


%investigar el tiempo de vida de los materiales organicos


%On the contrary, at low temperatures and/or in a rigid medium, phosphorescence can be observed.

%avientate algo de decir por que si se dio una transferencia de energiiia y vemos la emisooon de la rodamina, por que obtenemos valores grandes de sigmas ?



% A pesar de que se trabaj\'o con la soluci\'on de concentraci\'on mayor (Soluci\'on 1, c= $9.25 \times 10 ^{-6} M$), se mostr\'o con los espectros de absorci\'on y emisi\'on lineales, que se obtienen las mismas estructuras vesiculares que a menores concentraciones. 
%CONCLUSIONES -------^